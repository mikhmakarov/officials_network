\documentclass[a4paper,14pt]{article}

\usepackage{cmap}         % поиск в PDF
\usepackage{mathtext}         % русские буквы в формулах
\usepackage[T2A]{fontenc}     % кодировка
\usepackage[utf8]{inputenc}     % кодировка исходного текста
\usepackage[russian]{babel} % локализация и переносы
\usepackage{hyphenat}
\frenchspacing
\usepackage{amsmath,amsfonts,amssymb,amsthm,mathtools} % AMS
\usepackage{mathrsfs}
\usepackage{bbm}
\usepackage{icomma} % "Умная" запятая: $0,2$ --- число, $0, 2$ --- перечисление
\renewcommand{\leq}{\ensuremath{\leqslant}}
\renewcommand{\geq}{\ensuremath{\geqslant}}

\usepackage{graphicx}  % Для вставки рисунков
\setlength\fboxsep{3pt} % Отступ рамки \fbox{} от рисунка
\setlength\fboxrule{1pt} % Толщина линий рамки \fbox{}
\usepackage{indentfirst}
\usepackage{array,tabularx,tabulary,booktabs} % Дополнительная работа с таблицами
\usepackage{longtable}  % Длинные таблицы
\usepackage{multirow} % Слияние строк в таблице
\usepackage{extsizes} % Возможность сделать 14-й шрифт
\usepackage{geometry} % Простой способ задавать поля
\usepackage{setspace} % Интерлиньяж
\onehalfspacing % Интерлиньяж 1.5
\usepackage{lastpage} % Узнать, сколько всего страниц в документе.
\usepackage{hyperref}
\usepackage[usenames,dvipsnames,svgnames,table,rgb]{xcolor}
\usepackage{tabularx}
\usepackage{multicol} % Несколько колонок
\usepackage{listings}
\usepackage{etoolbox} % логические операторы
\usepackage{tikz} % Работа с графикой
\usepackage{cite} % Работа с библиографией
\usepackage{paralist}                   % compact lists
\usepackage{adjustbox} 
\addto\captionsrussian{% Replace "english" with the language you use
  \renewcommand{\contentsname}%
  {Оглавление}%
}
\usepackage{float}
\usepackage{polyglossia}
\setmainlanguage{russian} 
\setotherlanguage{english}
\newfontfamily\russianfont[Script=Cyrillic]{Linux Libertine}
\sloppy 
  
\usepackage{geometry} % Простой способ задавать поля
\geometry{verbose,tmargin=1in,bmargin=1in,lmargin=1in,rmargin=1in}

\author{Борис Цейтлин, Михаил Макаров}
\title{game theory in data analysis project}
\makeatother
\begin{document}



	\thispagestyle{empty}    % +1 - это титульный лист
	
	\begin{center}
		
		
		THE RUSSIAN GOVERNMENT \\
		FEDERAL STATE AUTONOMUS EDUCATIONAL INSTITUTION \\ FOR HIGHER PROFESSIONAL EDUCATION \\ NATIONAL RESEARCH UNIVERSITY \\ ''HIGHER SCHOOL OF ECONOMICS''
		
		\large
		\vspace{2 cm}
		%	\textsf{
		%	Факультет экономических наук
		%	\\ Образовательная программа «Экономика» %}           %ТЕКСТ БЕЗ ЗАСЕЧЕК
	\end{center}
	
	\vspace{2 cm}
	\begin{center}
		%	\vspace{13ex}
		\vspace{1 cm} \textbf{Игры и решения в задачах анализа данных и моделирования} \\ \vspace{0.5 cm} Project title
		
	\end{center}
	
	\vspace{2 cm}
	
	\begin{flushright}
		%	\noindent
		{  Борис Цейтлин \\ Михаил Макаров}
		
		\vspace{1 cm}
		
		{ \textbf{"Науки о данных" } \\ Факультет Компьютерных Наук }
	\end{flushright}
	
	\begin{center}
		\vfill
		Москва 2019
	\end{center}
	
	\newpage
  \tableofcontents
  \newpage

	\section{Введение}
	
  В данной работе мы проанализировали сеть контактов госслужащих РФ и выявили  корреляцию между центральностью госслужащего в сети и его доходами.

  Согласно № 273-ФЗ «О противодействии коррупции» государственные служащие РФ ежегодно подают декларации о доходах, имуществе и обязательствах имущественного характера. В декларациях указываются счета в банках, ценные бумаги, недвижимости, транспортные средства, доходы родственников и место работы.

  \subsection{Исходные данные}

  Данные были получены с помощью проекта \href{https://declarator.org}{Декларатор}, который собирает декларации государственных служащих, приводит их к машиночитаемому виду и предоставляет к ним доступ.

  \begin{figure}[H]
    \centering
    \adjustimage{max size={0.8\linewidth}{0.9\paperheight}}{img/declarator_example.png}
    \caption{Декларатор.орг}
  \end{figure}

  Проект предоставляет данные в формате JSON. 

  \begin{figure}[H]
    \centering
    \adjustimage{max size={0.4\linewidth}{0.5\paperheight}}{img/json_example.png}
    \caption{Пример данных в декларации}
  \end{figure}

  Всего дамп данных содержал \textbf{90864} деклараций, с 1998 года по 2017.
  Он включает в себя \textbf{51855} уникальных персоналий, \textbf{2104} учреждения.

  Мы выделили в декларациях интересующие нас параметры:
  \begin{itemize}
  \setlength{\itemindent}{.5in}
    \item person\_id, person\_name - идентификаторы человека
    \item year - год декларации
    \item office\_id, office\_name - идентификаторы учреждения
    \item income - доход в рублях
  \end{itemize}

  При этом доходы родственников мы добавляли к доходам государственных служащих.

  Мы обнаружили слишком много пропусков в данных о банковских вкладах, ценных бумагах, недвижимости и транспортных средствах. Эти данные остались вне области данного исследования.

  \pagebreak

  Таким образом, мы перешли к табличному виду данных следующего формата:

  \begin{figure}[H]
    \centering
    \adjustimage{max size={0.8\linewidth}{0.5\paperheight}}{img/data_table.png}
    \caption{Табличный вид данных}
  \end{figure}

  \section{Исследование данных}

  \subsection{Метаданные}

  \begin{figure}[H]
    \centering
    \adjustimage{max size={0.8\linewidth}{0.5\paperheight}}{img/total_dec.png}
    \caption{Количество деклараций по годам}
  \end{figure}

  Мы выделили, что основная часть деклараций приходится на года с 2009 по 2016. Декларации от других лет сильно отличаются от деклараций в этот период. Мы решили использовать только декларации за этот период для дальнейшего анализа.

  \begin{figure}[H]
    \centering
    \adjustimage{max size={1\linewidth}{0.5\paperheight}}{img/dec_by_office.png}
    \caption{Количество деклараций по учреждениям}
  \end{figure}

  \begin{figure}[H]
    \centering
    \adjustimage{max size={1\linewidth}{0.5\paperheight}}{img/unique_by_office.png}
    \caption{Количество персон по учреждениям}
  \end{figure}

  \begin{figure}[H]
    \centering
    \adjustimage{max size={1\linewidth}{0.7\paperheight}}{img/unique_by_year_by_office.png}
    \caption{Относительное количество персон в учреждениях по годам}
  \end{figure}

  Данная диаграмма показывает как менялось количество чиновников в разных учреждениях относительно общего числа.

  \subsection{Данные о доходах по годам и учреждениям}

  \begin{figure}[H]
    \centering
    \adjustimage{max size={1\linewidth}{0.7\paperheight}}{img/income_by_years.png}
    \caption{Суммарный заработок чиновников по годам}
  \end{figure}

  Примечательно, что суммарные доходы госслужащих сильно менялись от года к году. На данном графике явно выделяется экономический кризис 2011 - 2012 годов, который называют "второй волной" кризиса 2008 года.

  \begin{figure}[H]
    \centering
    \adjustimage{max size={1\linewidth}{0.7\paperheight}}{img/income_by_years_med.png}
    \caption{Медианный заработок чиновников по годам}
  \end{figure}

  Медианный доход однако менялся не сильно.

  \begin{figure}[H]
    \centering
    \adjustimage{max size={1\linewidth}{0.7\paperheight}}{img/office_incomes.png}
    \caption{Общий заработок по учреждениям}
  \end{figure}

  \begin{figure}[H]
    \centering
    \adjustimage{max size={1\linewidth}{0.7\paperheight}}{img/big_office_incomes.png}
    \caption{Общий заработок по учреждениям, только большие учреждения}
  \end{figure}

  Рассматривая распределение заработков госслужащих в больших, по количеству работников, учреждениях и в целом, видно, что госслужащие на высоких должностях, уровня губернатора и выше, имеют доходы сопоставимые с доходами всех работников больших учреждений.

  А так менялись медианные доходы в крупнейших учреждениях с течением времени:

  \begin{figure}[H]
    \centering
    \adjustimage{max size={1\linewidth}{0.7\paperheight}}{img/income_by_year_by_office_med.png}
    \caption{Медианные доходы по учреждениям}
  \end{figure}
  
  \subsection{Распределение доходов}

  Мы рассмотрели распределение суммарных доходов госслужащих.

  \begin{figure}[H]
    \centering
    \adjustimage{max size={1\linewidth}{0.7\paperheight}}{img/income_boxplot_1.png}
    \caption{Boxplot доходов госслужащих}
  \end{figure}

  При взгляде на эту диаграмму становится очевидно, что среди доходов чиновников существует огромное неравенство. Мы решили разбить доходы чиновников по перцентилям.

  \begin{figure}[H]
    \centering
    \adjustimage{max size={1\linewidth}{0.7\paperheight}}{img/perc_inc_log.png}
    \caption{Суммарные доходы чиновников по перцентилям, логарифмическая шкала}
  \end{figure}

  \begin{figure}[H]
    \centering
    \adjustimage{max size={1\linewidth}{0.7\paperheight}}{img/perc_am_log.png}
    \caption{Количество чиновников по перцентилям}
  \end{figure}

  Обратная зависимость количества людей и их доходов очевидна.

  \begin{figure}[H]
    \centering
    \adjustimage{max size={0.8\linewidth}{0.5\paperheight}}{img/boxplots_perc.png}
    \caption{Распределение доходов по перцентилям, млн. руб}

  \end{figure}

  Мы так же проверили, какая зависимость есть между количеством лет на госслужбе и доходом. Для этого пришлось удалить выбросы, оставив только доходы в пределах 3 стандартных отклонений.

  \begin{figure}[H]
    \centering
    \adjustimage{max size={0.8\linewidth}{0.5\paperheight}}{img/income_vs_years.png}
    \caption{Годы на госслужбе и доходы}
  \end{figure}

  \subsection{Выводы}

  В результате анализа мы пришли к выводу, что данные содержат множество выбросов и перед дальнейшим анализом необходимо их очистить. Мы решили оставить только лиц с доходом в пределах $3 \sigma$  отклонений, использовать декларации только с \textbf{2009} по \textbf{2016} годы.

  \section{Создание сети}

  \section{Исследование сети}

  \subsection{Параметры сети}

  \subsection{Распределение степеней вершин}

  \subsection{}

  \section{Центральности в сети и корреляции с доходом}

  \subsection{Классические индексы центральности}

  \section{Заключение}

	
\end{document}